
We begin our proof by
recalling the formula for $n$-order finite differences of function $f$
\begin{align*}
  \Delta^{n} f(x) = \sum_{k=0}^{n} \binom{n}{k} (-1)^{n-k} f(x+k).
\end{align*}
Now, we define the binomial function $F$, such that
\begin{align*}
  F(x) = \binom{r-sx}{n}.
\end{align*}
Thus, the $t$-order forward finite difference of $F$ is
\begin{align*}
  \Delta^{t} F(x) = \sum_{k=0}^{t} \binom{t}{k} \binom{r-s(x+k)}{n} (-1)^{n-k}
\end{align*}
We may notice that the equation above quite reminds the structure of
Knuth's binomial identity~\eqref{prop:knuth-identity}.
By evaluating $\Delta^{t} F(x)$ in zero, we reach our goal even further,
because
\begin{align*}
  \Delta^{t} F(0) = \sum_{k=0}^{t} \binom{t}{k} \binom{r-sk}{n} (-1)^{n-k}.
\end{align*}
Next, by setting $t=n$ gives the exact structure
of~\eqref{prop:knuth-identity}, such that
\begin{align*}
  \Delta^{n} F(0) = \sum_{k=0}^{n} \binom{n}{k} \binom{r-sk}{n} (-1)^{n-k}.
\end{align*}
Still, it may not be immediately clear why $\Delta^{n} F(x) = (-1)^n s^n$.
To make things work,
we refer to the formula (5.42) in Concrete mathematics,
see~\cite[p. 190]{graham1994concrete},
\begin{align*}
  \sum_{k} \binom{n}{k} (-1)^k \left[
    a_0 + a_1 k^1 + a_2 k^2 + \cdots + a_n k^n
  \right] = (-1)^n n! a_n,
\end{align*}
which states that $n$-th difference of a polynomial of degree $n$ in $k$ equals
to the coefficient of $k^n$ multiplied by $(-1)^n n!$.
This helps us a lot, because the binomial coefficient
$\binom{r-sk}{n}$ is actually a polynomial of degree $n$ in $r-sk$.
The famous identity in Stirling numbers of the first kind $\stirlingi{n}{k}$
shows it clearly

\begin{align*}
  \binom{r-sk}{n}
  &= \frac{(-1)^{0}}{n!} \stirlingi{n}{0} (r-sk)^0
  + \frac{(-1)^{1}}{n!} \stirlingi{n}{1} (r-sk)^1
  + \cdots
  + \frac{(-1)^{n}}{n!} \stirlingi{n}{n} (r-sk)^n \\
  &= \sum_{j=0}^{n} \frac{(-1)^{j}}{n!} \stirlingi{n}{j} (r-sk)^j.
\end{align*}
Thus, the coefficient $a_n$ of $k^n$ is
\begin{align*}
  a_n
  = [k^n] \frac{(-1)^{n}}{n!} \stirlingi{n}{n} (r-sk)^n
  = \frac{(-1)^{n}}{n!} \stirlingi{n}{n} [k^n] (r-sk)^n.
\end{align*}
By Binomial theorem,
\begin{align*}
  (r-sk)^n = \sum_{m=0}^{n} (-1)^{n-m} \binom{n}{m} r^{m} s^{n-m} k^{n-m}.
\end{align*}
Thus,
\begin{align*}
  [k^n] (r-sk)^n
  = (-1)^{n-0} \binom{n}{0} r^{0} s^{n-0}.
  = (-1)^{n} s^{n}
\end{align*}
Hence, the coefficient $a_n$ of $k^n$ equals to
\begin{align*}
  a_n = \frac{(-1)^{n}}{n!} \stirlingi{n}{n} (-1)^{n} s^{n}
  = \frac{s^{n}}{n!}.
\end{align*}
Which implies that,
\begin{align*}
  \Delta^{n} F(0)
  = \sum_{k=0}^{n} \binom{n}{k} \binom{r-sk}{n} (-1)^{n-k}
  = (-1)^n n! a_n
  = (-1)^n n! \frac{s^{n}}{n!}.
\end{align*}
Therefore, the Knuth's identity~\eqref{prop:knuth-identity} is indeed true
\begin{align*}
  s^n = (-1)^n \Delta^{n} F(0)
  = \sum_{k=0}^{n} \binom{n}{k} \binom{r-sk}{n} (-1)^{k}.
\end{align*}
This completes the proof of Proposition~\eqref{prop:knuth-identity}. \qed

It is quite remarkable that
although
the identity~\eqref{prop:knuth-identity} is a special case of
forward finite differences of $F(x) = \binom{r-sx}{n}$ evaluated
in zero $s^n = (-1)^n \Delta^{n} F(0)$ -- it holds for all $x$, because
the coefficient of $k^n$ remains $s^n$ for all $x$
\begin{proposition} [Generalized Knuths' binomial identity]
  \label{prop:generalized-knuths-identity}
  For non-negative integers $n$, and for arbitrary integers $s, r, x$
  \begin{align*}
    s^n = (-1)^n \Delta^{n} F(x)
    = \sum_{k=0}^{n} \binom{n}{k} \binom{r-sx-sk}{n} (-1)^{k}.
  \end{align*}
\end{proposition}
Because the coefficient of $k^n$ in $\binom{r-sx-sk}{n}$ is
\begin{align*}
  a_n
  = \frac{(-1)^{n}}{n!} \stirlingi{n}{n} [k^n] ((r-sx)-sk)^n
  = \frac{s^{n}}{n!}.
\end{align*}
Because, by Binomial theorem
\begin{align*}
  [k^n] ((r-sx)-sk)^n = (-1)^{n-0} \binom{n}{0} (r-sx)^{0} s^{n-0}
  = (-1)^n s^n.
\end{align*}
