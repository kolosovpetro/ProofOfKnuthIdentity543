
\begin{proposition}[Knuth binomial identity]
  \label{prop:knuth-identity}
  \begin{align*}
    \sum_{k=0}^{n} \binom{n}{k} \binom{r-sk}{n} (-1)^k = s^n
  \end{align*}
\end{proposition}

\begin{proposition}[Forward finite differences]
  \begin{align*}
    \Delta^{n} f(x) = \sum_{k=0}^{n} \binom{n}{k} (-1)^{n-k} f(x+k)
  \end{align*}
\end{proposition}

\begin{proposition} [Binomial function]
  \begin{align*}
    F(x) = \binom{r-sx}{n}
  \end{align*}
\end{proposition}

\begin{corollary}[Forward difference of Binomial function]
  \begin{align*}
    \Delta^{t} F(x) = \sum_{k=0}^{t} \binom{t}{k} (-1)^{n-k} \binom{r-s(x+k)}{n}
  \end{align*}
\end{corollary}
By setting $x=0$ in equation above yields
\begin{corollary}[Forward difference of Binomial function]
  \begin{align*}
    \Delta^{t} F(x) = \sum_{k=0}^{t} \binom{t}{k} \binom{r-sk}{n} (-1)^{n-k}
  \end{align*}
\end{corollary}
By setting $t=n$ gives the structure of~\eqref{prop:knuth-identity}, such that
\begin{corollary}[Forward difference of Binomial function]
  \begin{align*}
    \Delta^{n} F(x) = \sum_{k=0}^{n} \binom{n}{k} \binom{r-sk}{n} (-1)^{n-k}
  \end{align*}
\end{corollary}
Still, it is not quite clear why $\Delta^{n} F(x) = (-1)^n s^n$, until
we refer to relation given by Knuth in Concrete mathematics, see (5.42)
\begin{align*}
  \sum_{k} \binom{n}{k} (-1)^k \left[
    a_0 + a_1 k^1 + a_2 k^2 + \cdots + a_n k^n
  \right] = (-1)^n n! a_n
\end{align*}
meaning that $n$-th difference of a polynomial of degree $n$ in $k$ equals
to the coefficient of $k^n$ times $(-1)^n n!$.

Now we notice that $F(x) = \binom{r-sx}{n}$ is a polynomial of degree $n$ in $r-sk$
\begin{corollary}
  \begin{align*}
    \binom{r-sk}{n}
    &= \sum_{j=0}^{n} \frac{(-1)^{j}}{n!} \stirlingi{n}{j} (r-sk)^j \\
    &= \frac{(-1)^{0}}{n!} \stirlingi{n}{0} (r-sk)^0
    + \frac{(-1)^{1}}{n!} \stirlingi{n}{1} (r-sk)^1
    + \cdots
    + \frac{(-1)^{n}}{n!} \stirlingi{n}{n} (r-sk)^n
  \end{align*}
\end{corollary}
Thus, the coefficient of $k^n$ is
\begin{align*}
  a_n = [k^n] \frac{(-1)^{n}}{n!} \stirlingi{n}{n} (r-sk)^n
\end{align*}
By binomial theorem
\begin{align*}
  (r-sk)^n = \sum_{m=0}^{n} (-1)^{n-m} \binom{n}{m} r^{m} s^{n-m} k^{n-m}
\end{align*}
Thus,
\begin{align*}
  [k^n] (r-sk)^n
  = (-1)^{n-0} \binom{n}{0} r^{0} s^{n-0}
  = (-1)^{n} s^{n}
\end{align*}
Thus,
\begin{align*}
  a_n = \frac{(-1)^{n}}{n!} \stirlingi{n}{n} (-1)^{n} s^{n}
  = \frac{s^{n}}{n!}
\end{align*}
Thus,
\begin{align*}
  \Delta^{n} F(x)
  = \sum_{k=0}^{n} \binom{n}{k} \binom{r-sk}{n} (-1)^{n-k}
  = (-1)^n n! a_n
  = (-1)^n n! \frac{s^{n}}{n!}
\end{align*}
Hence, Knuth identity is indeed true
\begin{align*}
  s^n = (-1)^n \Delta^{n} F(x)
  = \sum_{k=0}^{n} \binom{n}{k} \binom{r-sk}{n} (-1)^{k}
\end{align*}
